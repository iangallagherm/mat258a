\documentclass[11pt]{article}

% Packages
\usepackage[utf8]{inputenc}
\usepackage{amsmath, amssymb, amsthm}
\usepackage{enumitem}
\usepackage{geometry}
\usepackage{fancyhdr}
\usepackage{mathtools}
\usepackage{multicol}
\usepackage{minted}
\usepackage{optidef}   % https://ctan.org/pkg/optidef
\usepackage[colorlinks=true, urlcolor=blue, linkcolor=blue, citecolor=blue]{hyperref}

% Page layout
\geometry{margin=0.75in}
\pagestyle{fancy}
\fancyhf{}
\rhead{Ian Gallagher}
\lhead{Math 207C Homework}
\rfoot{\thepage}
\setlength{\headheight}{14pt}

% Commands
\newcommand{\vep}{\varepsilon}
\DeclarePairedDelimiter\abs{\lvert}{\rvert}
\newcommand{\norm}[2]{\lVert #1 \rVert_{#2}}
\newcommand{\pvec}[1]{\begin{pmatrix}#1\end{pmatrix}}

% Environments
\newtheoremstyle{problemstyle}
  {1em} % Space above
  {1em} % Space below
  {\normalfont} % Body font
  {} % Indent amount
  {\bfseries} % Theorem head font
  {} % Punctuation after theorem head
  {\newline} % Space after theorem head
  {} % Theorem head spec

\theoremstyle{problemstyle}
\newtheorem{problem}{Problem}

% Custom commands
\newenvironment{solution}
  {\noindent\textbf{Solution}\quad}
  {\hfill$\blacksquare$\par\vspace{1em}}

% Enumerate styles
\setlist[enumerate,1]{label=(\alph*), ref=\alph*, itemsep=-0.2em, topsep=0.4em}
\setlist[enumerate,2]{label=(\roman*), ref=\roman*, itemsep=0em, topsep=0.2em}

% Title info
\title{Math 258A Challenge \#\texttt{5}}
\author{Ian Gallagher}
\date{\today}

\begin{document}

\maketitle

\section*{Problem 5}
Prove that given a finite collection of vertical segments in the plane, if for
every three segments there is a line intersecting them, there exists a line
intersecting them all.

\begin{proof}
  Let $S_1, S_2, \ldots, S_n$ be the finite collection of vertical segments in
  the the plane. For each $S_i$, consider the space of lines passing through
  that segment, parameterized by slope and y-intercept. In particular, define
  $L_i = \left\{(m, b) \in \mathbb{R}^2 : y = mx + b \text{ s.t } (x_i, mx_i +
  b) \in S_i\right\}$. The set $L_i$ is a closed, bounded, convex set. For
  closure, it must contain its limit points since the original line segment is
  closed. The line segment being finite length also guarantees that the set of
  legal slopes and y-intercepts must be bounded.

  To prove convexity, take two lines $y = m_1x + b_1$ and $y = m_2x + b_2$ in
  $L_i$. The line $y = (m_1x + b_1) + \theta \left(m_2x + b_2\right)$ ,$\theta
  \in [0, 1]$, is the line corresponding to the convex combination of the two
  points in the parameter space. We have $y(x_i) = y_1 + \theta(y_2 - y_1)$,
  where $y_1$ and $y_2$ are the y-coordinates of the two lines at $x_i$. Since
  $S_i$ is a convex line segment, this y-coordinate is in the segment.

  By the assumption that every three line segments, $S_i, S_j, S_k$ have a line
  intersecting them, we have that the intersection $L_{ijk} = L_i \cap L_j \cap
  L_k$ is non-empty. That is, the intersection of any three of the closed
  bounded convex sets is non-empty. By Helly's theorem, with $d = 2$, we have
  the intersection $L = \bigcap_{i=1}^n L_i$ is non-empty. Picking a point $(m, b)
  \in L$ gives the desired line $y = mx + b$ that intersects all the segments at
  once.
\end{proof}

\section*{Problem 13 (estimation of probability distribution)}
A random variable $\xi$ has possible values $\xi_1, \ldots, \xi_n$, but the
corresponding probabilities $p_1, \ldots, p_n$ are unknown. Formulate the
problem of finding $p_1, \ldots, p_n$ such that the variance of $\xi$ is
maximized, the expected value of $\xi$ is between $\alpha$ and $\beta$, the
probabilities sum to one and no probability is less than $0.01/n$. Reformulate
the resulting model as a minimization problem and check convexity.

\section*{Solution}
The maximization problem can be formulated as follows:
\begin{maxi}
  {p_1, \ldots p_n}
  {\operatorname{Var}(\xi) = \operatorname{E}[\xi^2] - \operatorname{E}[\xi]^2}
  {}{}
  \addConstraint{p_1 + \cdots + p_n}{= 1}
  \addConstraint{p_i}{\ge .01/n,\quad i=1,\dots,n}
  \addConstraint{\operatorname{E}[\xi]}{\ge \alpha}
  \addConstraint{\operatorname{E}[\xi]}{\le \beta}
\end{maxi}

\noindent The minimization problem is the same but with the objective function negated:
\begin{mini}
  {p_1, \ldots p_n}
  {\operatorname{E}[\xi]^2 - \operatorname{E}[\xi^2]}
  {}{}
  \addConstraint{p_1 + \cdots + p_n}{= 1}
  \addConstraint{\operatorname{E}[\xi]}{\ge \alpha}
  \addConstraint{\operatorname{E}[\xi]}{\le \beta}
  \addConstraint{p_i}{\ge .01/n,\quad i=1,\dots,n}
\end{mini}

\noindent The feasible region is the intersection of a set of affine equalities
and inequalities and is therefore convex. The objective function is a positive
quadratic function in the variables $p_i$ and is therefore convex. Therefore,
this minization problem is convex.

\end{document}

