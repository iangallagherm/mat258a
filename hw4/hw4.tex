\documentclass[11pt]{article}

% Packages
\usepackage[utf8]{inputenc}
\usepackage{amsmath, amssymb, amsthm}
\usepackage{enumitem}
\usepackage{geometry}
\usepackage{fancyhdr}
\usepackage{mathtools}
\usepackage{multicol}
\usepackage{minted}
\usepackage[colorlinks=true, urlcolor=blue, linkcolor=blue, citecolor=blue]{hyperref}

% Page layout
\geometry{margin=0.75in}
\pagestyle{fancy}
\fancyhf{}
\rhead{Ian Gallagher}
\lhead{Math 207C Homework}
\rfoot{\thepage}
\setlength{\headheight}{14pt}

% Commands
\newcommand{\vep}{\varepsilon}
\DeclarePairedDelimiter\abs{\lvert}{\rvert}
\newcommand{\norm}[2]{\lVert #1 \rVert_{#2}}
\newcommand{\pvec}[1]{\begin{pmatrix}#1\end{pmatrix}}

% Environments
\newtheoremstyle{problemstyle}
  {1em} % Space above
  {1em} % Space below
  {\normalfont} % Body font
  {} % Indent amount
  {\bfseries} % Theorem head font
  {} % Punctuation after theorem head
  {\newline} % Space after theorem head
  {} % Theorem head spec

\theoremstyle{problemstyle}
\newtheorem{problem}{Problem}

% Custom commands
\newenvironment{solution}
  {\noindent\textbf{Solution}\quad}
  {\hfill$\blacksquare$\par\vspace{1em}}

% Enumerate styles
\setlist[enumerate,1]{label=(\alph*), ref=\alph*, itemsep=-0.2em, topsep=0.4em}
\setlist[enumerate,2]{label=(\roman*), ref=\roman*, itemsep=0em, topsep=0.2em}

% Title info
\title{Math 258A Challenge \#\texttt{4}}
\author{Ian Gallagher}
\date{\today}

\begin{document}

\maketitle

\section*{Problem 2}
The problem of finding the point on the parabola
\( y = \tfrac15\,(x-1)^2 \) closest to \( (1,2) \) in the Euclidean sense
is a constrained optimization problem with one constraint.
\begin{enumerate}[label=(\alph*)]
  \item Find all the points that satisfy the KKT conditions.
        Is LICQ satisfied?  Which of the points are optimal solutions?

  \item Are the second‑order necessary conditions satisfied?
        How about the second‑order sufficient conditions?
        What is its Lagrange dual problem?

  \item You may be tempted to substitute the single constraint in the
        objective function and eliminate one variable to get an
        unconstrained optimization problem.
        Can the solutions of these problems be solutions to the original one?

  \item We saw in class the LICQ (and a couple other CQs too)
        as a technical assumption to prove KKT.
        Show an example where the active constraints are linear
        (so KKT holds!) yet the LICQ is not satisfied.
\end{enumerate}

\section*{Solution}

% --------------------------------------------------------------------------
\subsection*{Points that satisfy the KKT conditions}
We want to minimize the function $f(x,y) = \tfrac{1}{2}\left[(x-1)^2 +
(y-2)^2\right]$ subject to the constraint $g(x,y) = y - \tfrac{1}{5}(x-1)^2 =
0$. Then the Lagrangian is
\[
  \mathcal{L}(x,y,\lambda) = f(x,y) + \lambda g(x,y)
\]
where $\lambda$ is the Lagrange multiplier and can be positive or negative for
the equality constraint. Checking the optimality condition, we have
\[
  \nabla_{(x,y)} \mathcal{L}(x,y,\lambda) = \nabla f(x,y) + \lambda \nabla g(x,y)
  = \pvec{x-1\\y-2} + \lambda \pvec{-2/5(x-1)\\1} = \mathbf{0}
\]
This gives the two equations $(x-1)(1 - \tfrac{2}{5}\lambda) = 0$ and $y - 2 +
\lambda = 0$. The first equation gives us two cases:
\begin{enumerate}
  \item Case 1: $x = 1$.
        Then the second equation gives $y = 2 - \lambda$.
        The constraint gives $2 - \lambda = \tfrac{1}{5}(1-1)^2 = 0$,
        so $\lambda = 2$ and $(x,y) = (1,0)$.

  \item Case 2: $1 - \tfrac{2}{5}\lambda = 0$.
        Then $\lambda = \tfrac{5}{2}$ and $y = 2 - \tfrac{5}{2} =
        -\tfrac{1}{2}$. This immediately fails since the constraint $g(x,y) = 0$
        gives $y = \tfrac{1}{5}(x-1)^2 \geq 0$.
\end{enumerate}
\noindent Thus the only point that satisfies the KKT conditions is $(x^*,y^*) =
(1,0)$ with $\lambda^* = 2$.

\noindent LICQ is satisfied as well since our only active constraint has $\nabla g =
(0,1)^T \neq 0$. 

% --------------------------------------------------------------------------
\subsection*{Second order necessary and sufficient conditions}
The necessary condition for optimality is that the Hessian of the Lagrangian is
positive semidefinite. The sufficient condition is that the Hessian is positive
definite. The Hessian of the Lagrangian is
\[
  \nabla^2 \mathcal{L}(x,y,\lambda) = \pvec{2 -\frac{2}{5} \lambda & 0\\0 & 2}
\]
Substituting in $(x^*,y^*,\lambda^*) = (1,0,2)$ gives the diagonal matrix with
entries $\frac{1}{5}$ and $2$. Both eigenvalues are positive, so $\nabla^2
\mathcal{L} \succ 0$ and we may conclude that the point is optimal.

\subsubsection*{Lagrange Dual Problem}
The dual function is defined as $d(\lambda) = \inf_{(x,y)}
\mathcal{L}(x,y,\lambda)$. To get the infimum over the Lagrangian, we can
substitute the zeros from the gradient KKT conditions $x = 1$ and $y = 2 -
\lambda$ to get
\[
  d(\lambda) = \mathcal{L}(1,2-\lambda,\lambda) = \tfrac{1}{2}\lambda^2 +
  \lambda(2 - \lambda) = 2\lambda - \tfrac{1}{2}\lambda^2 = \tfrac{\lambda}{2}(4
  - \lambda)
\]
Maximizing this gives the dual problem of the given optimization problem. Since
it is a downward facing quadratic with roots at $0, 4$, the maximum is at
$\lambda^* = 2$ and $d(\lambda^*) = 4$. Strong duality is therefore attained.

% --------------------------------------------------------------------------
\subsection*{Substituting constraints}
The issue of substituting the constraint into the objective function is
dependent on which substitution is made (for the given problem). If we
substitute the given formula for $y$, the objective function becomes:
\[
  f(x, \tfrac{1}{5}(x-1)^2) = \tfrac{1}{50}\left[(x-1)^4 + 5(x-1)^2 + 100\right]
\]
Since we have positive powers of $(x-1)$, the function is minimized at $x^* = 1$
and this corresponds to the previous solution.

However, if we write $x = 1 \pm \sqrt{5y}$, we get
\[
  f(1 \pm \sqrt{5y}, y) = \tfrac{1}{2}\left[y^2 + y + 4\right]
\]
This quadratic has a minimum at $y^* = -\tfrac{1}{2}$, which is not a feasible
solution of the original problem. Therefore, we can not naively expect solutions
of unconstrained formulations to be solutions of the original problem.
% --------------------------------------------------------------------------
\subsection*{Failure of LICQ}
A somewhat contrived example of a problem with active constraints that are
linear is the following:
\[
\begin{alignedat}{2}
\text{minimize}  \quad & x+y   \\
\text{subject to}\quad & x = 0 \\
                       & y = 0 \\
                       & x + y = 0
\end{alignedat}
\]
The only feasible point (and also the optimal point) is $(0,0)$, and it is
simple to check that the KKT conditions are satisfied. However, LICQ fails since
the gradient vectors of the active constraints are linearly dependent. For
example, we have the following linear combination of the gradients
\[
  \pvec{1\\0} + \pvec{0\\1} = \pvec{1\\1}.
\]

\end{document}

