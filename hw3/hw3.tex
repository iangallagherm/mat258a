\documentclass[11pt]{article}

% Packages
\usepackage[utf8]{inputenc}
\usepackage{amsmath, amssymb, amsthm}
\usepackage{enumitem}
\usepackage{geometry}
\usepackage{fancyhdr}
\usepackage{mathtools}
\usepackage{multicol}
\usepackage{minted}
\usepackage[colorlinks=true, urlcolor=blue, linkcolor=blue, citecolor=blue]{hyperref}

% Page layout
\geometry{margin=0.75in}
\pagestyle{fancy}
\fancyhf{}
\rhead{Ian Gallagher}
\lhead{Math 207C Homework}
\rfoot{\thepage}
\setlength{\headheight}{14pt}

% Commands
\newcommand{\vep}{\varepsilon}
\DeclarePairedDelimiter\abs{\lvert}{\rvert}
\newcommand{\norm}[2]{\lVert #1 \rVert_{#2}}

% Environments
\newtheoremstyle{problemstyle}
  {1em} % Space above
  {1em} % Space below
  {\normalfont} % Body font
  {} % Indent amount
  {\bfseries} % Theorem head font
  {} % Punctuation after theorem head
  {\newline} % Space after theorem head
  {} % Theorem head spec

\theoremstyle{problemstyle}
\newtheorem{problem}{Problem}

% Custom commands
\newenvironment{solution}
  {\noindent\textbf{Solution}\quad}
  {\hfill$\blacksquare$\par\vspace{1em}}

% Enumerate styles
\setlist[enumerate,1]{label=(\alph*), ref=\alph*, itemsep=-0.2em, topsep=0.4em}
\setlist[enumerate,2]{label=(\roman*), ref=\roman*, itemsep=0em, topsep=0.2em}

% column vector in parentheses
\newcommand{\pvec}[1]{\begin{pmatrix}#1\end{pmatrix}}

% Title info
\title{Math 258A Challenge \#\texttt{3}}
\author{Ian Gallagher}
\date{\today}

\begin{document}

\maketitle

\noindent (Worked jointly with Santiago Morales before writing up individual
solutions).

\section*{Problem 1}
Consider the nonlinear optimization problem

\[
\text{max} \quad \frac{x_1}{x_2 + 1}
\]
\[
\text{\bf subject to:} \quad x_1 - x_2 \leq 2, \quad x_1 \geq 0, \quad x_2 \geq 0.
\]

% --------------------------------------------------------------------------
\subsection*{Use the KKT conditions to show that (4,2) is not optimal}

The objective function and constraints for the given optimization problem are
\begin{align*}
  f(x) = \frac{x_1}{x_2+1}, \\
  g_1(x)=x_1-x_2-2\le0, \\
  g_2(x)=-x_1\le0, \\
  g_3(x)=-x_2\le0 .
\end{align*}
The Lagrangian is therefore
\[
  L(x,\lambda)=f(x)-\sum_{i=1}^3\lambda_i g_i(x)
\]
with $\lambda_i\ge0$.

Checking for slack variables, we find that $g_1(4,2)=0$ while $g_2(4,2)=-4$,
$g_3(r,2)=-2$. Complementarity conditions therefore require that $\lambda_2 =
\lambda_3 = 0$. Checking the gradient condition, we have

\begin{align*}
\nabla f(4,2)-\lambda_1\nabla g_1(4,2)=0 \\
(1/3,\,-4/9)-\lambda_1(1,-1)=0.
\end{align*}
Since $(1/3, -4/9)$ is not a scalar multiple of $(1,-1)$, there is no possible
$\lambda_1$ that satisfies the equation. We may conclude that the point $(4,2)$
is not optimal.

% --------------------------------------------------------------------------
\subsection*{A solution that does satisfy the KKT conditions}

To maximize the rational function $f$ requires making the numerator as large as
possible while simultaneously minimizing the denominator. This pushes any
potential optimal solution onto the right boundary of the feasible region. The
point $(2,0)$ is a good guess since it will allow two degrees of freedom when
evaluating the Lagrangian gradient condition.

That is, \((2,0)\) satisfies \(g_1(x)=g_3(x)=0\) and \(g_2(x)=-2<0\). So
$\lambda_2=0$ and we have
\[
  \nabla f(2,0)-\lambda_1\nabla g_1(x) -\lambda_3\nabla g_3(x)
  =(1,-2)-\lambda_1(1,-1)-\lambda_2(0,-1) = 0
\]
Choosing $\lambda_1=1$ and $\lambda_3=1$ gives the desired equality, and this
suffices to show that $(2,0)$ satisfies the KKT conditions.

% --------------------------------------------------------------------------
\subsection*{Show this problem is not convex}

Since we are maximizing our objective function, convexity will require that $f$
be concave and that the feasible region is convex. The feasible region here is
an intersection of linear inequalities, which is convex. The problem must be
with the objective function.
Take the points \(A=(0,2)\) and \(B=(2,0)\) inside the feasible region. Their
midpoint is \(C=(1,1)\). Evaluating the objective function at these points gives
\begin{align*}
  f(A)&=\frac{0}{3}=0,\\
  f(B)&=\frac{2}{1}=2,\\
  f(C)&=\frac{1}{2}=0.5.
\end{align*}
Concavity would demand
\(f(C)\ge\frac12f(A)+\frac12f(B)=1\), but we have shown this is not the case.
Therefore, the given problem is not convex.

% --------------------------------------------------------------------------
\subsection*{Optimality}

As we have previously argued, any optimal solution must lie on the right
boundary of the feasible region, where $x_1-x_2=2$. Substituting this into the
objective gives
\[
f(x)=\frac{x_2+2}{x_2+1}=1+\frac{1}{x_2+1}.
\]
This is a decreasing function of $x_2$ for $x_2\ge0$, so the maximum must occur
at $x_2^*=0$. The optimal solution is therefore $(x_1^*,x_2^*)=(2,0)$ and we are
done.

\end{document}

